\documentclass[12pt,twoside]{article}
\usepackage{graphicx}
\usepackage{epsfig}
\usepackage{fancyhdr}
\usepackage{subeqnarray}
%\usepackage{subfigure}
%\usepackage{subfig}
\usepackage{subcaption}
\captionsetup[subtable]{position=bottom}
\usepackage{dsfont}
\usepackage[T1]{fontenc}
\usepackage{ae,aecompl}
\usepackage{pdfpages}
% can use backref=page
\usepackage[]{hyperref}
\hypersetup{
    colorlinks,
    citecolor=black,
    filecolor=black,
    linkcolor=black,
    urlcolor=black
}
\usepackage{caption}
%\usepackage{color}
\usepackage{transparent}
\usepackage{titlesec}
\usepackage{grffile}
% \usepackage{comment}
% \excludecomment{figure}
% \let\endfigure\relax
\usepackage{placeins}
\usepackage{attachfile}
\usepackage{bbold}
\usepackage{amsmath}
\usepackage{booktabs}
\usepackage{array}
\usepackage{dcolumn}
\usepackage{textcomp}
\usepackage{lscape}
%\usepackage[strings]{underscore}
%\usepackage[normalem]{ulem}
\usepackage{soul}

% This is FULLPAGE.STY by H.Partl, Version 2 as of 15 Dec 1988.
% Document Style Option to fill the paper just like Plain TeX.

%\typeout{Style Option FULLPAGE Version 2 as of 15 Dec 1988}

\topmargin 0pt
\advance \topmargin by -\headheight
\advance \topmargin by -\headsep

\textheight 8.9in

\oddsidemargin 0pt
\evensidemargin \oddsidemargin
\marginparwidth 0.5in

\textwidth 6.5in


% For users of A4 paper: The above values are suited for american 8.5x11in
% paper. If your output driver performs a conversion for A4 paper, keep
% those values. If your output driver conforms to the TeX standard (1in/1in),
% then you should add the following commands to center the text on A4 paper:

 \advance\hoffset by -3mm  % A4 is narrower.
 \advance\voffset by  8mm  % A4 is taller.

%\endinput
%\renewcommand{\chaptername}{}
% \newcommand{\mychapter}[2]{
%     \setcounter{chapter}{#1}
%     \setcounter{section}{0}
%     \chapter*{#2}
%     \addcontentsline{toc}{chapter}{#2}

\titleformat{\chapter}
  {\Large\bfseries} % format
  {}                % label
  {0pt}             % sep
  {\huge}           % before-code    

 
\newcolumntype{d}[1]{D{.}{.}{#1}}
\newcommand\mc[1]{\multicolumn{1}{c}{#1}} % handy shortcut macro

\begin{document}

\thispagestyle{empty}

\bigskip

% \begin{abstract}

% \end{abstract}
\title{Log file for Barrier Option}

\maketitle
\thispagestyle{empty}

\pagestyle{fancy}
\fancyhf{}
\renewcommand{\headrulewidth}{0pt}
\fancyhead[RO,LE]{\thepage}
\fancyhead[LO,RE]{}
\fancyfoot[CO,CE]{}
\fancyfoot[RO]{
\begin{picture}(0,0)
%\put(-15,0){\includegraphics[scale=0.5]{}}
\put(0,-10){\includegraphics[scale=0.15]{files/GMT_logo_blue}}
\end{picture}
}
\fancyfoot[RE]{
\begin{picture}(0,0)
%\put(-60,0){\includegraphics[scale=0.5]{}}
\put(0,-10){\includegraphics[scale=0.15]{files/GMT_logo_blue}}
\end{picture}
}

\clearpage
\tableofcontents
\thispagestyle{empty}

\newpage
\pagenumbering{arabic}

% \maketitle
% 
% \begin{abstract}
% 
% \end{abstract}

\section{Introduction}

The goal is to express all barrier option payoffs in terms of a unique ``atom'' formula. Rebates are not included at the moment.

\section{Atom: ``up-and-out; down-and-out''}

The fundamental payoff of choice is relative to the ``up-and-out; down-and-out'' payoff. The barriers are $L, U$ and the active window if the continuously monitored interval $[t_s, t_e]$. $T_e$ is the expiry of the option, $S$ the underlying and $X$ the strike. 

\begin{eqnarray}
 \text{payoff} & = & \max(S-X, 0)\ \mathbb{1}_{\forall t \in [t_s, t_e]: L < S(t) < U}\ \nonumber \\
               & \equiv\ & \text{uodo}(S, X, L, U, \{t_{s}, t_e\}, T_{e})
\end{eqnarray}

\section{The zoo}


\subsection{vanilla}

\begin{eqnarray}
 \text{payoff} & = & \max(S-X, 0) \nonumber \\
               & = & \text{uodo}(S, X, 0, \infty, \{0, T_{e}\}, T_{e})
\end{eqnarray}


\subsection{single barrier, B}

\noindent \underline{``up-and-out'' - $S_0 < B$}

\begin{eqnarray}
 \text{payoff} & = & \max(S - X, 0) \mathbb{1}_{\forall t \in [t_s, t_e]: S(t) < B}\ \nonumber \\
               & = & \text{uodo}(S, X, 0, B, \{t_{s}, t_e\}, T_{e})
\end{eqnarray}

\noindent \underline{``down-and-out'' - $B < S_0$}

\begin{eqnarray}
 \text{payoff} & = & \max(S - X, 0) \mathbb{1}_{\forall t \in [t_s, t_e]: B < S(t)}\ \nonumber \\
               & = & \text{uodo}(S, X, B, \infty, \{t_{s}, t_e\}, T_{e})
\end{eqnarray}

\noindent \underline{``down and in'' - $ B < S_0$}

\begin{eqnarray}
 \text{payoff} & = & \max(S-X, 0) \mathbb{1}_{\exists t^{\ast} \in [t_s, t_e]: S(t^{\ast}) < B}\ \nonumber \\
               & = & \max(S-X, 0) \left(1 - \mathbb{1}_{\forall t \in [t_s, t_e]: B < S(t)} \right)\ \nonumber \\
               & = & \text{``vanilla'' - ``down-and-out''} \nonumber \\
               & = & \text{uodo}(S, X, 0, \infty, \{0, T_{e}\}, T_{e}) - \text{uodo}(S, X, B, \infty, \{t_{s}, t_e\}, T_{e})
\end{eqnarray}

\noindent \underline{``up and in'' - $ S_0 < B$}

\begin{eqnarray}
 \text{payoff} & = & \max(S-X, 0) \mathbb{1}_{\exists t^{\ast} \in [t_s, t_e]: B < S(t^{\ast})}\ \nonumber \\
               & = & \max(S-X, 0) \left(1 - \mathbb{1}_{\forall t \in [t_s, t_e]: S(t) < B} \right)\ \nonumber \\
               & = & \text{``vanilla'' - ``up-and-out''} \nonumber \\
               & = & \text{uodo}(S, X, 0, \infty, \{0, T_{e}\}, T_{e}) - \text{uodo}(S, X, 0, B, \{t_{s}, t_e\}, T_{e})
\end{eqnarray}


\subsection{double barrier, \{L, U\}}

\noindent \underline{``double knock-in'' - $L < S_{0} < U$}

\begin{eqnarray}
 \text{payoff} & = & \max(S-X, 0)\ \mathbb{1}_{\exists t^{\ast} \in [t_s, t_e]: S(t^{\ast}) < L || U < S(t^{\ast})} \nonumber \\
               & = & \max(S-X, 0)\ \left(1 -  \mathbb{1}_{\forall t \in [t_s,  t_e]: L < S(t) < U}\right) \nonumber \\
               & = & \text{``vanilla'' - ``up-and-out;down-and-out''} \nonumber \\
               & = & \text{uodo}(S, X, 0, \infty, \{0, T_{e}\}, T_{e}) - \text{uodo}(S, X, L, U, \{t_{s}, t_e\}, T_{e})
\end{eqnarray}


\bibliography{./files/references_tmp.bib}
\bibliographystyle{./files/gmt_tech_report}


\end{document}
